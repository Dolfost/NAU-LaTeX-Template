\scriptsize
\footnotesize
\small
\normalsize
\large
\Large
\LARGE
\huge
\Huge


\newcommand{command}[arguments][optional]{definition}
command
The name of the new command, starting with a backslash followed by lowercase and/or uppercase letters or a backslash followed by a single non-letter symbol. That name must not be already defined and is not allowed to begin with \end.

arguments
An integer from 1 to 9, the number of arguments of the new command. If omitted, the command will have no arguments.

optional
If this is present, then the first of the arguments would be optional with a default value given here. Otherwise all arguments are mandatory.

definition
Every occurrence of the command will be replaced by definition and every occurrence of the form #n will then be replaced by the nth argument.


\parbox[alignment]{width}{text}
\parbox[alignment][height][inner alignment]{width}{text}
\begin{minipage}{width}
\end{minipage}

\footnote{text}

\hyphenation{acro-nym}
\hyphenation{ac-ro-nym ac-ro-nym-ic a-cro-nym-i-cal-ly}
\mbox{indivisible}.

\\[value] %would insert additional vertical space after the break depending on the value, like \\[3mm].

The command \linebreak has a direct counterpart: \nolinebreak. This command prevents a line break at the current position. Like its counterpart, it takes an optional argument. If you write \nolinebreak[0], you recommend to not break the line there. Using 1, 2, or even 3 makes the request stronger and \nolinebreak[4] forbids it completely. The latter will be presumed if you don't provide an argument.

\enlargethispage{\baselineskip} 
There's a starred version: \enlargethispage* would additionally shrink all vertical spaces on the page to their minimum.


\includegraphics[height=10cm,width=10cm,scale=1,angle=0]{imagefile}

There are starred forms of floats, namely, figure* and table*. In a two-column layout, they put the float into a single column. In one-column mode, there's no difference to the non-starred form.

 \begin{wrapfigure}[number of lines]{placement}[overhang] {width}
 	
 	\addcontentsline{toc}{chapter}{Preface}
 	
 	\addtocontents{toc}{\bigskip}
 	\addcontentsline{toc}{part}{Appendix}
 	
 	\part				-1  (book and report class)
 	\chapter		   	 0  (not available in article class)
 	\section			 1
 	\subsection			 2
 	\subsubsection		 3
 	\paragraph			 4
 	\subparagraph		 5

‹ \href{URL}{text} makes text to a hyperlink, which points to the URL address
‹ \url{URL} prints the URL and links it
‹ \nolinkurl{URL} prints the URL without linking it
‹ \hyperref{label}{text} changes text to a hyperlink, which links to the place where the label has been set, thus to the same place \ref{label} would point to
‹ \hypertarget{name}{text} creates a target name for potential hyperlinks with text as the anchor
‹ \hyperlink{name}{text} makes text to a hyperlink, which points to the target name


\section{The equation \texorpdfstring{$y=x^2$}{y=x\texttwosuperior}}