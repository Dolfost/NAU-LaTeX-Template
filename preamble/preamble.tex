% !TeX root = ../homework.tex
\documentclass[12pt]{article}
\usepackage[english, ukrainian]{babel}
\usepackage[utf8]{inputenc}
\usepackage[T1]{fontenc}


\usepackage{geometry}
\geometry{
	paper=a4paper,
	inner=2cm, outer=2cm, top=2cm, bottom=2cm, 
	bindingoffset=0cm, 
	textwidth=\textwidth, textheight=\textheight, 
	includehead=false, includefoot=false, portrait, twoside}
\usepackage{layouts}
\usepackage{graphicx}
\usepackage[dvipsnames]{xcolor}
\usepackage{blindtext}


%\usepackage{parskip} 									% remove the paragraph indentation


\usepackage{amsmath}
\numberwithin{equation}{section}
\usepackage{amsthm}
\usepackage{thmtools}
\declaretheoremstyle[
notefont=\normalfont\itshape, notebraces={}{},
headfont=\bfseries\sffamily,
bodyfont=\normalfont,
headformat=\NAME\ \NUMBER \NOTE,
headpunct=\sffamily.\\,
spaceabove=2mm, spacebelow=2mm,
postheadspace=0mm, headindent=3mm,
]{thmstyle}
\declaretheorem[style=thmstyle, numberwithin=section, name=Теорема]{thm}
\declaretheorem[style=thmstyle, numberwithin=section, name=Означення]{dfn}
\usepackage{amsfonts}
\usepackage{amssymb}
\usepackage{mathtools}
\mathtoolsset{showonlyrefs=false}
\usepackage{esvect} % \vv{n} gives the n vector


\usepackage{array}
\setlength{\extrarowheight}{1.3mm}
\usepackage{booktabs}
%\setlength{\abovetopsep}{1cm}
\usepackage{multirow}
\usepackage{tabularx}
\usepackage{nicematrix}
\NiceMatrixOptions{cell-space-top-limit=3pt}
\usepackage{longtable}
\setcounter{LTchunksize}{40} % default: 20


\usepackage{float}
\usepackage{wrapfig}
\usepackage{caption}
\captionsetup{font=sf, labelfont=sf, margin=1cm, skip=2.5mm}
\usepackage{titlesec}
\titleformat{\section}{\LARGE\normalfont\bfseries\sffamily}{\thesection.}{2.2mm}{}
\titleformat{\subsection}{\Large\normalfont\bfseries\sffamily}{\thesubsection.}{2.2mm}{}
\titleformat{\paragraph}{\large\normalfont\bfseries\sffamily}{}{}{}
\titleformat{\subparagraph}{\normalsize\normalfont\bfseries\sffamily}{}{2mm}{}
\titlespacing*{\section}{0mm}{5mm}{1mm}
\titlespacing*{\subsection}{0mm}{5mm}{1mm}
\titlespacing*{\paragraph}{0mm}{2mm}{1mm}
\titlespacing*{\subparagraph}{0mm}{2mm}{1mm}


\usepackage{fancyhdr}
\fancyhf{}
\renewcommand{\headrulewidth}{0.18mm}
\renewcommand{\footrulewidth}{0mm}
\fancyhead[L]{\scriptsize\sffamily\hyperlink{toc}{ПЕРЕЙТИ ДО ЗМІСТУ}}
\fancyhead[RO,RE]{\sffamily\leftmark} 
\fancyfoot[CE,CO]{\rule[0.18mm]{3cm}{0.1mm} .: {\ttfamily\bfseries\thepage}\ :. \rule[0.18mm]{3cm}{0.1mm}}
\fancyfoot[R]{\sffamily\shortfaculty}
\fancyfoot[L]{\sffamily\shortdepartment\ \shortuniversity}
\pagestyle{fancy}


\usepackage{tcolorbox}
\usepackage[cachedir=minted_package_cache]{minted}
\usemintedstyle{tango}
\setminted{numbers=none, autogobble, breaklines, bgcolor=CodeBg, frame=bottomline, framerule=0.4pt, framesep=0.1mm,  highlightcolor=LimeGreen, curlyquotes=true, mathescape}
\BeforeBeginEnvironment{minted}{}
\AfterEndEnvironment{minted}{}
%\newminted{cpp}{linenos,autogobble,breaklines, bgcolor=white}

\usepackage{enumitem}
\setlist{noitemsep, itemsep=0.6mm, topsep=0.3cm, labelsep=2mm}%, nolistsep}
\setitemize{label=\(\mathbin{\Diamond}\), font=\ttfamily, align=right}
\setitemize[2]{label=\textopenbullet}
\setitemize[3]{label=\(\vartriangleleft\)}
\setenumerate{label=\textcircled{\scriptsize\Alph*}, font=\sffamily, align=right, start=1}
\setenumerate[2]{label=\textcircled{\scriptsize\arabic*}}
\setdescription{style = nextline, font=\bfseries, align=left}
\setdescription[2]{font=\itshape}


\usepackage{xspace}


\usepackage{microtype}


\usepackage[singlespacing]{setspace}


\usepackage[nottoc]{tocbibind}
\usepackage{xifthen}
\newif\iffigures
\AtBeginEnvironment{figure}{%
	\iffigures\else\global\figurestrue\fi
}
\newif\iftables
\AtBeginEnvironment{table}{%
	\iftables\else\global\tablestrue\fi
}
\AtBeginEnvironment{longtable}{%
	\iftables\else\global\tablestrue\fi
}

\newif\iflistings
\AtBeginEnvironment{listing}{%
	\iflistings\else\global\listingstrue\fi
}
\newif\iftheorems
\AtBeginEnvironment{thm}{%
	\iftheorems\else\global\theoremstrue\fi
}
\newif\ifdefenitions
\AtBeginEnvironment{dfn}{%
\ifdefenitions\else\global\defenitionstrue\fi
}


\usepackage{biblatex}
\addbibresource{\subjectpath/bibliography.bib}
\usepackage{imakeidx}
\indexsetup{firstpagestyle=fancy}
\newcommand{\mainindex}[1]{{\bfseries #1}}
\makeindex[options=-s iso]


\usepackage[]{hyperref}
\definecolor{linkcolorus}{RGB}{80,80,80}
\hypersetup{
	colorlinks=true,
	linkcolor=linkcolorus,
	raiselinks=true,
	backref=true,
	pdfview=XYZ,
	pdfviewarea=TrimBox,
	linktocpage=true,
	anchorcolor=black,
	citecolor=green,
	filecolor=cyan,
	urlcolor=magenta,
	frenchlinks=false, % Uses small caps for links instead of color
	bookmarks=true, % Writes bookmarks for the Acrobat Reader
	bookmarksopen=false % Shows all bookmarks in an expanded view
	bookmarksnumbered=false % Includes the section number in bookmarks
	pdfstartpage=1 % Specifies which page would be shown when the PDF file is opened
}
\hypersetup{
	pdfauthor={\studentname},
	pdfcreator={Created with LaTex by \studentname},
	pdftitle={\prodname},
	pdfsubject={},
	pdfkeywords={equations,mathematics,programming}
}
\usepackage{cleveref}
\crefname{figure}{рис.}{рис.}
\Crefname{figure}{Рис.}{Рис.}
\crefname{table}{табл.}{табл.}
\Crefname{table}{Табл.}{Табл.}
\crefname{listing}{ліс.}{ліс.}
\Crefname{listing}{Ліс.}{Ліс.}

\Crefname{section}{Розд.}{Розд.}
\Crefname{section}{розд.}{розд.}
\crefname{page}{стр.}{стр.}

\crefname{equation}{вир.}{вир.}
\Crefname{equation}{Вир.}{Вир.}
\crefname{thm}{трм.}{трм.}
\Crefname{thm}{Трм.}{Трм.}
\crefname{dfn}{озн.}{озн.}
\Crefname{dfn}{Озн.}{Озн.}
\usepackage{nameref}


\renewcommand{\footnoterule}
{\noindent\smash{\rule[1.5mm]{0.5\textwidth}{0.3pt}}}

\setcounter{tocdepth}{4}

\frenchspacing


%%% 			Preamble made by Vladyslav Rehan			  %%%
% 	https://tex.stackexchange.com/users/283011/vladyslav-rehan	%
% 	https://github.com/Dolfost									%