% !TeX root = ../document.tex
\documentclass[14pt]{report}
\usepackage[english, ukrainian]{babel}
\usepackage[utf8]{inputenc}
\usepackage[T1]{fontenc}

\usepackage{geometry}
\geometry{%
	paper=a4paper,
	inner=1cm, outer=1cm, top=2cm, bottom=2cm, 
	bindingoffset=1.5cm, 
	textwidth=\textwidth, textheight=\textheight, 
	includehead=false, includefoot=false, portrait}
\usepackage{layouts}
\usepackage{graphicx}
\usepackage[dvipsnames]{xcolor}
\usepackage{blindtext}

% \usepackage{parskip} % remove the paragraph indentation


\usepackage{textcomp}


\usepackage[section]{placeins}


\usepackage[titletoc, toc, title]{appendix}


\usepackage{amsmath}
\numberwithin{equation}{section}
\usepackage{amsthm}
\usepackage{thmtools}
\declaretheoremstyle[
notefont=\normalfont\itshape, notebraces={}{},
headfont=\bfseries\sffamily,
bodyfont=\normalfont,
headformat=\NAME\ \NUMBER\ \NOTE,
headpunct=\sffamily.\\,
spaceabove=2mm, spacebelow=2mm,
postheadspace=0mm, headindent=3mm,
]{thmstyle}
\declaretheorem[style=thmstyle, numberwithin=section, name=Теорема]{thm}
\declaretheorem[style=thmstyle, numberwithin=section, name=Означення]{dfn}
\usepackage{amsfonts}
\usepackage{amssymb}
\usepackage{mathtools}
\mathtoolsset{showonlyrefs=false}
\usepackage{esvect} % \vv{n} gives the n vector


\usepackage{array}
\setlength{\extrarowheight}{1.3mm}
\usepackage{booktabs}
\usepackage{changepage}
%\setlength{\abovetopsep}{1cm}
\usepackage{multirow}


\usepackage{tabularx}
\usepackage{nicematrix}
\NiceMatrixOptions{cell-space-top-limit=3pt}
\usepackage{longtable}
\setcounter{LTchunksize}{40} % default: 20


\usepackage{float}
\usepackage{wrapfig}
\usepackage{caption}
\counterwithin{figure}{chapter}
\captionsetup{font=sf, labelfont=sf, margin=1cm, skip=2.5mm, labelsep=endash, figureposition=below,
			  tableposition=above}
			  \DeclareCaptionFormat{tableCaption}{{#1#2#3}}
\captionsetup[table]{format=tableCaption, justification=raggedright, singlelinecheck=false, position=top, name={Таблиця}, labelsep=endash}
\usepackage{titlesec}
\titleclass{\chapter}{straight}
\titleformat{\chapter}{\huge\normalfont\bfseries\sffamily}{\thechapter.}{2.2mm}{}
\titleformat{\section}{\LARGE\normalfont\bfseries\sffamily}{\thesection}{2.2mm}{}
\titleformat{\subsection}{\Large\normalfont\bfseries\sffamily}{\thesubsection}{2.2mm}{}
\titleformat{\subsubsection}{\large\normalfont\bfseries\sffamily}{\thesubsubsection}{2.2mm}{}
\titleformat{\paragraph}{\large\normalfont\bfseries\sffamily}{\theparagraph}{}{}
\titleformat{\subparagraph}{\large\normalfont\bfseries\sffamily}{\thesubparagraph}{2mm}{}
\titlespacing*{\chapter}{\parindent}{4mm}{1mm}
\titlespacing*{\section}{\parindent}{4mm}{1mm}
\titlespacing*{\subsection}{\parindent}{4mm}{1mm}
\titlespacing*{\subsubsection}{\parindent}{3mm}{1mm}
\titlespacing*{\paragraph}{\parindent}{2mm}{1mm}
\titlespacing*{\subparagraph}{\parindent}{2mm}{1mm}


\usepackage{fancyhdr}
\fancyhf{}
\renewcommand{\headrulewidth}{0.18mm}
\renewcommand{\footrulewidth}{0mm}
% \fancyhead[L]{\scriptsize\sffamily\hyperlink{tocLink}{ПЕРЕЙТИ ДО ЗМІСТУ}}
\fancyhead[RO,RE]{\sffamily\leftmark} 
\fancyfoot[CE,CO]{.:\ {\ttfamily\bfseries\thepage}\ :.}
\fancyfoot[R]{\sffamily\shortfaculty}
\fancyfoot[L]{\sffamily\shortdepartment\ \shortuniversity}
\pagestyle{fancy}


\usepackage{tcolorbox}
\definecolor{CodeBackground}{RGB}{235,235,235}
\usepackage[cachedir=minted_package_cache]{minted}
\newenvironment{longlisting}{\captionsetup{type=listing}}{}
\usemintedstyle{tango}
\setminted{numbers=right, linenos=true, autogobble, breaklines, tabsize=4,
bgcolor=CodeBackground, frame=leftline, framerule=0.35pt, framesep=2mm,
numbersep=1mm, highlightcolor=LimeGreen, curlyquotes=true, mathescape,
stepnumber=1, fontsize=\footnotesize, xleftmargin=0cm}

\usepackage{enumitem}
\setlist{noitemsep, itemsep=1mm, topsep=1mm, labelsep=2mm}%, nolistsep}
\setlist[itemize,1]{label=\textbullet, font=\ttfamily, align=right}
\setlist[itemize, 2]{label=\textopenbullet}
\setlist[itemize, 3]{label=\(\vartriangleleft\)}
% \setlist[enumerate, 1]{, align=right, start=1}
% \setlist[enumerate, 2]{label=\textcircled{\scriptsize\Alph*}}
% \setlist[enumerate, 3]{label=\textcircled{\scriptsize\Roman*}}
\setlist[enumerate, 1]{label=\arabic*., font=\sffamily, align=right, start=1}
\setlist[enumerate, 2]{label*=\arabic*.}
\setlist[enumerate, 3]{label*=\arabic*.}
\setlist[description, 1]{itemsep=3mm, style=nextline, font=\mdseries, align=left}
\setlist[description, 2]{style=nextline, font=\mdseries, align=left}
\newlist{steps}{enumerate}{1}
\setlist[steps,1]{label={Крок \arabic*.}}
\newlist{guiDescription}{enumerate}{3}
\setlist[guiDescription, 1]{label={\tt\arabic*.}, ref={\tt\arabic*}, font=\sffamily, start=1}
\setlist[guiDescription, 2]{label={\Alph*}}
\setlist[guiDescription, 3]{label={\Roman*}}


\usepackage{xspace}


\usepackage{microtype}


% \usepackage[singlespacing]{setspace}


\usepackage[nottoc]{tocbibind}
\usepackage{xifthen}
\newif\iffigures
\AtBeginEnvironment{figure}{%
	\iffigures\else\global\figurestrue\fi
}
\newif\iftables
\AtBeginEnvironment{table}{%
	\iftables\else\global\tablestrue\fi
}
\AtBeginEnvironment{longtable}{%
	\iftables\else\global\tablestrue\fi
}

\newif\iflistings
\AtBeginEnvironment{listing}{%
	\iflistings\else\global\listingstrue\fi
}
\newif\iftheorems
\AtBeginEnvironment{thm}{%
	\iftheorems\else\global\theoremstrue\fi
}
\newif\ifdefenitions
\AtBeginEnvironment{dfn}{%
\ifdefenitions\else\global\defenitionstrue\fi
}


\usepackage[sorting=none]{biblatex}
\addbibresource{\subjectpath/bibliography.bib}

\usepackage{imakeidx}
\indexsetup{firstpagestyle=fancy}
\newcommand{\mainindex}[1]{{\bfseries #1}}
\makeindex[options=-s iso]


\usepackage[]{hyperref}
\definecolor{linkcolorus}{RGB}{80,80,80}
\hypersetup{
	colorlinks=true,
	linkcolor=linkcolorus,
	citecolor=linkcolorus,
	raiselinks=true,
	backref=true,
	pdfview=XYZ,
	pdfviewarea=TrimBox,
	linktocpage=true,
	anchorcolor=black,
	filecolor=cyan,
	urlcolor=magenta,
	frenchlinks=false, % Uses small caps for links instead of color
	bookmarks=true, % Writes bookmarks for the Acrobat Reader
	bookmarksopen=false % Shows all bookmarks in an expanded view
	bookmarksnumbered=false % Includes the section number in bookmarks
	pdfstartpage=1 % Specifies which page would be shown when the PDF file is opened
}
\hypersetup{
	pdfauthor={\studentname},
	pdfcreator={Created with LaTex by \studentname},
	pdftitle={\prodname},
	pdfsubject={},
	pdfkeywords={equations,mathematics,programming}
}
\usepackage[ukrainian]{cleveref}
% \crefname{figure}{рис.}{рис.}
% \Crefname{figure}{Рис.}{Рис.}
% \crefname{table}{табл.}{табл.}
% \Crefname{table}{Табл.}{Табл.}
% \crefname{listing}{ліс.}{ліс.}
% \Crefname{listing}{Ліс.}{Ліс.}

% \Crefname{section}{Розд.}{Розд.}
% \Crefname{section}{розд.}{розд.}
% \crefname{page}{стр.}{стр.}

% \crefname{equation}{вир.}{вир.}
% \Crefname{equation}{Вир.}{Вир.}
% \crefname{thm}{трм.}{трм.}
% \Crefname{thm}{Трм.}{Трм.}
% \crefname{dfn}{озн.}{озн.}
% \Crefname{dfn}{Озн.}{Озн.}
% \renewcommand{\crefandconjunction}{ і }
\usepackage{nameref}

\renewcommand{\footnoterule}
{\noindent\smash{\rule[1.5mm]{0.5\textwidth}{0.3pt}}}

\setcounter{tocdepth}{5}
\setcounter{secnumdepth}{3} % Set the depth for numbering paragraphs

\frenchspacing

% add image path to the image
% \LetLtxMacro{\OldIncludegraphics}{\includegraphics}
% \renewcommand{\includegraphics}[2][]{%
%     \OldIncludegraphics[#1]{#2}%
% 	\vspace{-2cm}%
% 	\center{\smash{\colorbox{red}{\textcolor{white}{\tiny \tt \detokenize{#2}}}}}%
% 	\vspace{2cm}%
% }


%%% 			Preamble made by Vladyslav Rehan			  %%%
% 	https://tex.stackexchange.com/users/283011/vladyslav-rehan	%
% 	https://github.com/Dolfost									%