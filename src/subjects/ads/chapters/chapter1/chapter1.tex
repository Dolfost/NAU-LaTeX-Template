\documentclass[\mainDocument]{subfiles}

\begin{document}
	\nocite{*}

	\chapter{Тема і мета роботи}
	\bigtext{Тема:} Тема\\
	\bigtext{Мета:} Мета
	\chapter{Постановка задачі}
	\begin{enumerate}
		\item
	\end{enumerate}
	Варіант 5 (варіант \(=N\mod7\) де \(N=12\))

	\chapter{Теоретичні відомості}
	Вміст цієї секції взято з~\cite{Karumanchi_2020}.

	\chapter{Тестування}
	\section{Про програму}
	Програма для тестування алгоритмів написана на швидкій і компактній скриптовій мові
	програмуванні \textinline{Lua}.
	\section{Структура}
	\begin{description}
		\item[\textinline{algorithms.lua}] Функції алгоритмів Shell sort, Insertion sort.
		\item[\textinline{io.lua}] Функція для друкування масивів.
		\item[\textinline{options.lua}] Опис термінальних опцій та їхніх властивостей.
		\item[\textinline{random.lua}] Функція для генерування масивів фіксованої довжини з елементами в заданому проміжку.
		\item[\textinline{sort.lua}] Головний файл програми де всі вищезгадані методи виконуються.
	\end{description}
	\subsection{Опції}
	Програма має широкий набір опцій.
	\section{Запуск і перевірка алгоритмів}

	\chapter{Висновки}
	\begin{itemize}
		\item Ознайомився:
			\begin{itemize}
				\item
			\end{itemize}
		\item Вивчив:
			\begin{itemize}
				\item
			\end{itemize}
		\item Реалізував:
			\begin{itemize}
				\item
			\end{itemize}
	\end{itemize}
\end{document}