%TeX root = ../../../document.tex
\sffamily

\section{Постановка задачі}
Вміст цієї секції взятий з \cite{CSHPiskunov}.\\
\bigtext{Тема:}\\
\bigtext{Мета:}
\begin{itemize}
	\item
\end{itemize}

\paragraph{Вимоги}
Зразок вимог взятий з \cite{kulikov}. 
\begin{description}
	\item[Системні характеристики]\directenv
		\begin{enumerate}
			\item Програма має бути консольним додатком
			\item Програма розробляється мовою програмування \code{C\#} 
			\item Програма не обов'язково має бути кросс платформенною
		\end{enumerate}
	\item[Вимоги користувача]\directenv
		\begin{enumerate}
			\item Запуск і зупинка додатку (відбувається в терміналі)
				\begin{description}
					\item[Windows]
						Запуск відбувається за допомогою префіксу відносного шляху \code{.\textbackslash}
				\end{description}
			\item Налаштування програми
				\begin{enumerate}
					\item Налаштування відбувається через передачу слів як аргументів командного рядка
				\end{enumerate}
			\item Перегляд журналу роботи
				\begin{enumerate}
					\item Журнал роботи не передбачений
				\end{enumerate}
		\end{enumerate}
	\item[Атрибути якості]\directenv
		\begin{enumerate}
			\item Стійкість до вхідних даних
				\begin{enumerate}
					\item Неякісні вхідні дані не мають приводити до аварійного завершення програми
				\end{enumerate}
		\end{enumerate}
	\item[Детальні специфікації]\directenv
		\begin{enumerate}
			\item Версія \code{dotnet framework} - 4.0
		\end{enumerate}
\end{description}

\section{Теорія}
\subsection{Про \code{C\#}}

\section{Про програму}

\section{Тестування}
\begin{figure}[H]
	\centering
	\includegraphics[width=\textwidth]{images/test}
	\caption{Тестування}
	\label{fig:tag}
\end{figure}

\addtocontents{lof}{\bigskip}
\section{Висновки}
\begin{itemize}
	\item Ознайомився:
		\begin{itemize}
			\item З створенням діалогових вікон у .\code{Net}.
		\end{itemize}
	\item Вивчив:
		\begin{itemize}
		\end{itemize}
	\item Реалізував:
		\begin{itemize}
		\end{itemize}
\end{itemize}