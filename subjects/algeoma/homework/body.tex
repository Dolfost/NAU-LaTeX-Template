% !TeX root = ../../../homework.tex
\section{Знайти НСД}
Знайти \(\gcd(f(x),\ g(x))\) за допомогою алгоритму Евкліда
\begin{gather}
	f(x)=^4+x^3-3x^2-4x-1,\ g(x)=x^3-x^2-x-1.\label{eq:q:1:1}\\
	f(x)=x^4-x^3-4x+5,\ g(x)=x^3-x^2-2x+1.\label{eq:q:1:2}
\end{gather}
\subsection{НСД \eqref{eq:q:1:1}}
\solving
Знайдемо НСД функцій \(f(x)\) та \(g(x)\) \eqref{eq:q:1:1} за допомогою алгоритму Евкліда (довге ділення дільника на остачу):
\begin{gather}
	\begin{array}{r}
		x+2\phantom{)}   \\
		d_1(x)=x^3-x^2-x-1{\overline{\smash{\big)}\,x^4+x^3-3x^2-4x-1\phantom{)}}}\\
		\underline{-~\phantom{(}(x^4-x^3-x^2-x)\phantom{-b)}}\\
		2x^3-2x^2-3x-1\phantom{)}\\ 
		\underline{-~\phantom{()}(2x^3-2x^2-2x-2)}\\ 
		r_1(x)=-x+1\phantom{)}
	\end{array}
\end{gather}
Розділимо \(d_1(x)\) на \(r_1(x)\):
\begin{gather}
\begin{array}{r}
	x^2-2x\phantom{)}   \\
	d_2(x)=x+1{\overline{\smash{\big)}\,x^3-x^2-x\phantom{)}}}\\
	\underline{-~\phantom{(}x^3+x^2\phantom{-b)}}\\
	-2x^2-x\phantom{)}\\ 
	\underline{-~\phantom{()}-2x^2-2x}\\ 
	r_2(x)=x\phantom{)}
\end{array}
\end{gather}
Розділимо \(d_2(x)\) на \(r_2(x)\):
\begin{gather}
\begin{array}{r}
	1\phantom{)}   \\
	d_3(x)=x{\overline{\smash{\big)}\,x+1\phantom{)}}}\\
	\underline{-~\phantom{(}x\phantom{-b)}}\\
	r_3(x)=1\phantom{)}\\ 
\end{array}
\end{gather}
Розділимо \(d_3(x)\) на \(r_3(x)\):
\begin{gather}
	\begin{array}{r}
		x\phantom{)}   \\
		d_3(x)=1{\overline{\smash{\big)}\,x\phantom{)}}}\\
		\underline{-~\phantom{(}x\phantom{-b)}}\\
		r_3(x)=0\phantom{)}\\ 
	\end{array}
\end{gather}
Так як \(r_3(x)=0\Rightarrow d_3(x)\) буде НСД для \(f(x) \text{ та } g(x)\). Функції \(f(x)\) та \(g(x)\) взаємно прості.
\ansver
\begin{gather}
	\gcd(f(x),g(x))=1.
\end{gather}
\subsection{НСД \eqref{eq:q:1:2}}
\solving
Знайдемо НСД функцій \(f(x)\) та \(g(x)\) \eqref{eq:q:1:1} за допомогою алгоритму Евкліда (довге ділення дільника на остачу):
\begin{gather}
		\begin{array}{r}
		\frac{1}{2}x-\frac{1}{4}\phantom{)}   \\
		d_1(x)=2x^3-x^2-2x+1{\overline{\smash{\big)}\,x^4-x^3-4x+5\phantom{)}}}\\
		\underline{-~\phantom{(}(x^4-\frac{x^3}{2}-x^2+\frac{1}{2}x)\phantom{-b)}}\\
		-\frac{x^3}{2}+x^2-4.5x+5\phantom{)}\\ 
		\underline{-~\phantom{()}(-\frac{x^3}{2}+\frac{x^2}{4}+\frac{x}{2}-\frac{1}{4})}\\ 
		r_1(x)=\frac{3x^2}{4}-5x+5.25\phantom{)}
	\end{array}
\end{gather}
Розділимо \(d_1(x)\) на \(r_1(x)\):
\begin{gather}
	\begin{array}{r}
		\frac{8}{3}x+\frac{4\cdot17}{9}\phantom{)}   \\
		d_2(x)=\frac{3x^2}{4}-5x+5.25{\overline{\smash{\big)}\,2x^3-\frac{20}{3}x^2+14x\phantom{)}}}\\
		\underline{-~\phantom{(}2x^3-\frac{20}{3}x^2+14x\phantom{-b)}}\\
		-~\phantom{(}\frac{17}{3}x^2-16x+1\phantom{-b)}\\
		\underline{\frac{17}{3}x^2\-\frac{20\cdot17}{9}x+39\frac{2}{3}\phantom{)}}\\ 
		r_2(x)=182\frac{1}{3}x-38\frac{2}{3}\phantom{)}
	\end{array}
\end{gather}
Розділимо \(d_2(x)\) на \(r_2(x)\):
\begin{gather}
	\begin{array}{r}
		x\frac{3^2}{547\cdot4}-\frac{2648}{547^2}\phantom{)}   \\
		d_3(x)=182\frac{1}{3}x-38\frac{2}{3}{\overline{\smash{\big)}\,\frac{3x^2}{4}-5x+5.25\phantom{)}}}\\
		\underline{-~\phantom{(}\frac{3}{4}x^2-\frac{87}{547}x\phantom{-b)}}\\
		-~\phantom{(}-4\frac{460}{547}x+5.25\phantom{-b)}\\
		\underline{~\phantom{(}-4\frac{460}{547}x+\frac{307168}{897627}\phantom{-b)}}\\
		\dots
	\end{array}
\end{gather}
\ansver\dots\footnote{Я кілька раз перераховував і в мене виходили безглузді дроби\dots}

\newpage
\section{Використовуючи схему Горнера, знайти...}
Використовуючи схему Горнера, знайти:
\begin{itemize}
	\item частку та залишок від ділення могочлена \eqref{q:2:1} на \((x-2)\)
	\item розклад многочлена \eqref{q:2:1} за степенями \((x-2)\)
\end{itemize}
\begin{gather}
	f(x)=3x^6-4x^5-4x^4+2x^3-3x^2+2x-3.\label{q:2:1}
\end{gather}
\subsection{Частка та залишок}
\solving
Розділимо \eqref{q:2:1} на \((x-2)\) методом Горнера:
\begin{table}[H]
	\centering
	\caption{Ділення \(f(x)\) на \((x-2)\)}
	\begin{tabular}{|c|c|c|c|c|c|c|c|}
		\hline
		\(x_0\)&\(x^6\)&\(x^5\)&\(x^4\)&\(x^3\)&\(x^2\)&\(x^1\)&\(x^0\)\\
		\hline
		2&3&-4&-4&2&-3&2&-3\\
		\hline
		2&3&\(-4+3\cdot2=2\)&\(-4+4=0\)&\(2+0=2\)&\(-3+4=1\)&\(2+2=4\)&\(-3+8=5\)\\
		\hline
	\end{tabular}
\label{tab:1}
\end{table}
Згідно \cref{tab:1}
\begin{gather}
	s(x)=3x^5+2x^4+2x^2+x+4,\\
	r(x)=5,
\end{gather}
та \(f(x)\) можна записати як
\begin{gather}
	f(x)=(x-2)s(x)+r(x)=(x-2)(3x^5+2x^4+2x^2+x+4)+5.
\end{gather}
\ansver
\begin{gather}
	s(x)=3x^5+2x^4+2x^2+x+4,\\
	r(x)=5.
\end{gather}
\newpage
\subsection{Розклад за степенями}
\solving
Розкладемо \eqref{q:2:1} за степенями \((x-2)\) методом Горнера. Запишемо загальну форму \(f(x)\) степенями \((x-2)\) з невідомими коефіцієнтами:
\begin{gather}
	f(x)=A\left(x-2\right)^{6}+B\left(x-2\right)^{5}+C\left(x-2\right)^{4}+D\left(x-2\right)^{3}+E\left(x-2\right)^{2}+F\left(x-2\right)+G.\label{eq:2}
\end{gather}
Знайдемо невідомі коефіцієнти методом Горнера. Для зручності, проведемо обрахунки в таблиці \ref{tab:2}.
\begin{table}[H]
	\centering
	\caption{Розклад \(f(x)\) за степенями \((x-2)\)}
	\begin{NiceTabular}{>{\(}c<{\)}*{8}{|>{\(}c<{\)}}}[corners=SE,hvlines]
	x_0&x^6&x^5&x^4&x^3&x^2&x^1&x^0&\\
%	\hline
	2&3&-4&-4&2&-3&2&-3&X_i\\
%	\hline
	2&3&-4+6=2&-4+4=0&2+0=2&1&4&-3+8=5&G\\
%	\hline
	2&3&2+6=8&0+16=16&2+32=34&69&143&F\\
%	\hline
	2&3&14&44&122&313&E\\
%	\hline
	2&3&14+6=20&44+40=84&290&D\\
%	\hline
	2&3&20+6=26&136&C\\
%	\hline
	2&3&26+6=32&B\\
%	\hline
	2&3&A
	\end{NiceTabular}
	\label{tab:2}
\end{table}
Згідно \cref{tab:2} та \cref{eq:2}, \(f(x)\) можна записати як
\begin{multline}
	f(x)=3\left(x-2\right)^{6}+32\left(x-2\right)^{5}+136\left(x-2\right)^{4}+\\+290\left(x-2\right)^{3}+313\left(x-2\right)^{2}+142\left(x-2\right)+5.
\end{multline}
\ansver
\begin{multline}
	f(x)=3\left(x-2\right)^{6}+32\left(x-2\right)^{5}+136\left(x-2\right)^{4}+\\+290\left(x-2\right)^{3}+313\left(x-2\right)^{2}+142\left(x-2\right)+5.
\end{multline}