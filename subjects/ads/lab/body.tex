% !TeX root = ../../../document.tex
\nocite{*}
\sffamily

\section{Тема і мета роботи}
\bigtext{Тема:} \\
\bigtext{Мета:} \\
\section{Постановка задачі}
\begin{enumerate}
	\item
\end{enumerate}
Варіант 5 (варіант \(=N\mod7\) де \(N=12\))

\section{Теоретичні відомості}
Вміст цієї секції взято з \cite{Karumanchi_2020}.

\section{Тестування}
\subsection{Про програму}
Програма для тестування алгоритмів написана на швидкій і компактній скриптовій мові
програмуванні \code{Lua}.
\paragraph{Структура}
\begin{description}
	\item[\code{algorithms.lua}] Функції алгоритмів Shell sort, Insertion sort.
	\item[\code{io.lua}] Функція для друкування масивів.
	\item[\code{options.lua}] Опис термінальних опцій та їхніх властивостей.
	\item[\code{random.lua}] Функція для генерування масивів фіксованої довжини з елементами в заданому проміжку.
	\item[\code{sort.lua}] Головний файл програми де всі вищезгадані методи виконуються.
\end{description}
\paragraph{Опції}
Програма має широкий набір опцій.
\subsection{Запуск і перевірка алгоритмів}

\section{Висновки}
\begin{itemize}
	\item Ознайомився:
		\begin{itemize}
			\item
		\end{itemize}
	\item Вивчив:
		\begin{itemize}
			\item
		\end{itemize}
	\item Реалізував:
		\begin{itemize}
			\iteg
		\end{itemize}
\end{itemize}