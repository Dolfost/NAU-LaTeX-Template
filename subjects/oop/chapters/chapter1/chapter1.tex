\documentclass[../../../../document]{subfiles}

\begin{document}
	\chapter{Постановка задачі}
	Вміст цієї секції взятий з~\cite{CSHPiskunov}.\\
	\bigtext{Тема:}\\
	\bigtext{Мета:}
	\begin{itemize}
		\item
		\item
	\end{itemize}

	\input{\subfix{requirements-section}}

	\chapter{Теорія}
	Програма написана на \textinline|C#| з застосуванням Windows Forms (WinForms).

	Windows Forms --- це безкоштовна бібліотека графічних (GUI) класів з
	відкритим вихідним кодом, що входить до складу Microsoft.NET, .NET Framework
	або Mono, надаючи платформу для написання клієнтських додатків для настільних,
	портативних і планшетних комп'ютерів.
	\section{Про \texttt{C\#}}
	Вміст цієї секції взятий з~\cite{mslearn}.
	\begin{funcDescription}
		\funcitem[]{cs}||
		
	\end{funcDescription}

	\chapter{Про програму}

	\begin{figure}[htb]
		\centering
		% \includegraphics[width=\textwidth]{images/class.drawio}
		\caption{UML-діаграма класів}%
		\label{fig:classdiagram}%
	\end{figure}

	\chapter{Тестування}
	\begin{figure}[htb]
		\centering
		% \includegraphics[width=.93\textwidth]{images/test1}
		\caption{Тестування}%
		\label{fig:tag}
	\end{figure}

	\addtocontents{lof}{\bigskip}
	\chapter{Висновки}
	\begin{itemize}
		\item Ознайомився:
			\begin{itemize}
				\item
				\item
			\end{itemize}
		\item Реалізував:
			\begin{itemize}
				\item
			\end{itemize}
	\end{itemize}
\end{document}