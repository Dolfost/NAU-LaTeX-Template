\documentclass[../../../../document]{subfiles}

\begin{document}
	\chapter{Постановка задачі}
	Вміст цієї секції взятий з~\cite{CSHPiskunov}.\\
	\bigtext{Тема:}\\
	\bigtext{Мета:}
	\begin{itemize}
		\item
		\item
	\end{itemize}

	\section{Вимоги}
	Зразок вимог взятий з~\cite{kulikov}. 
	\begin{description}
		\item[Системні характеристики]\directenv%
			\begin{enumerate}
				\item Програма має бути консольним додатком
				\item Програма розробляється мовою програмування \code{C\#} 
				\item Програма не обов'язково має бути кросс платформенною
			\end{enumerate}
		\item[Вимоги користувача]\directenv%
			\begin{enumerate}
				\item Запуск і зупинка додатку (відбувається в терміналі)
					\begin{description}
						\item[Windows]
							Запуск відбувається за допомогою префіксу відносного шляху \code{.\textbackslash}
						\item[Unix-like]
							Програма не працює на даній платформі
					\end{description}
				\item Налаштування програми
					\begin{enumerate}
						\item Налаштування відбувається через передачу слів як аргументів командного рядка
					\end{enumerate}
				\item Перегляд журналу роботи
					\begin{enumerate}
						\item Журнал роботи не передбачений
					\end{enumerate}
			\end{enumerate}
		\item[Атрибути якості]\directenv%
			\begin{enumerate}
				\item Стійкість до вхідних даних
					\begin{enumerate}
						\item Неякісні вхідні дані не мають приводити до аварійного завершення програми
					\end{enumerate}
				\item Вікна
					\begin{enumerate}
						\item Вікно не повинно мати елементів \enquote{приховати} та \enquote{на повний екран}
						\item Вікно не має мати можливості змінювати свій розмір
						\item Вікно має бути білим.
						\item Класс вікна \code{inWnd} це наслідування класу \code{OkCancel}.
						\item Вікно має мати панель інструментів.
						\item Кнопка About у панелі інструментів має показувати вміст AssemblyInfo.
					\end{enumerate}
			\end{enumerate}
		\item[Детальні специфікації]\directenv%
			\begin{enumerate}
				\item Версія \code{dotnet framework} -- 4.0
			\end{enumerate}
	\end{description}


	\chapter{Теорія}
	Програма написана на \code{C\#} з застосуванням Windows Forms (WinForms).

	Windows Forms --- це безкоштовна бібліотека графічних (GUI) класів з
	відкритим вихідним кодом, що входить до складу Microsoft.NET, .NET Framework
	або Mono, надаючи платформу для написання клієнтських додатків для настільних,
	портативних і планшетних комп'ютерів.
	\section{Про \texorpdfstring{\code{C\#}}{C\#}}
	Вміст цієї секції взятий з~\cite{mslearn}.
	\begin{description}
		\item[\code{}]
	\end{description}

	\chapter{Про програму}

	\begin{figure}[htb]
		\centering
		% \includegraphics[width=\textwidth]{images/class.drawio}
		\caption{UML-діаграма класів}
		\label{fig:classdiagram}
	\end{figure}

	\chapter{Тестування}
	\begin{figure}[htb]
		\centering
		% \includegraphics[width=.93\textwidth]{images/test1}
		\caption{Тестування}
		\label{fig:tag}
	\end{figure}

	\addtocontents{lof}{\bigskip}
	\chapter{Висновки}
	\begin{itemize}
		\item Ознайомився:
			\begin{itemize}
				\item
				\item
			\end{itemize}
		\item Реалізував:
			\begin{itemize}
				\item
			\end{itemize}
	\end{itemize}
\end{document}