\documentclass[../../../../document]{subfiles}

\begin{document}
	\chapter{Постановка задачі}
	Вміст цієї секції взятий з~\cite{computational_methods}.\\
	\bigtext{Тема:}\\
	\bigtext{Мета:}
	\begin{itemize}
		\item
		\item
	\end{itemize}

	\chapter{Теорія}
	\chapter{Про програму}
	Програма написана на \textinline|C++| з застосуванням Qt framework.

	Qt (вимовляється як \enquote{к'ют}) - це крос-платформне
	програмне забезпечення для створення графічних інтерфейсів користувача, а також
	крос-платформних додатків, які працюють на різних програмних і апаратних
	платформах, таких як Linux, Windows, macOS, Android або вбудованих системах, з
	невеликими змінами або без змін у базовій кодовій базі, залишаючись при цьому
	нативним додатком з нативними можливостями і швидкістю.
	\section{Про \texttt{C++}}
	\begin{funcDescription}
		\funcitem{cs}|int a f(double b)|
		
	\end{funcDescription}
	


	\begin{figure}[htb]
		\centering
		% \includegraphics[width=\textwidth]{class.drawio}
		\caption{Блок-схема функції \codeline{int a(int b)}}%
		\label{fig:blockdiagram}%
	\end{figure}

	\chapter{Тестування}
	\begin{figure}[htb]
		\centering
		% \includegraphics[width=.93\textwidth]{test1}
		\caption{Тестування}%
		\label{fig:tag}
	\end{figure}

	\chapter{Висновки}
	\begin{itemize}
		\item Ознайомився:
			\begin{itemize}
				\item
				\item
			\end{itemize}
		\item Реалізував:
			\begin{itemize}
				\item
			\end{itemize}
	\end{itemize}
\end{document}
