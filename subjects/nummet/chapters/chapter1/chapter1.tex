\documentclass[../../../../document]{subfiles}

\begin{document}
	\chapter{Постановка задачі}
	Вміст цієї секції взятий з~\cite{computational_methods}.\\
	\bigtext{Тема:} \worktheme.\\
	\bigtext{Мета:}
	\begin{itemize}
		\item
		\item
	\end{itemize}
	Варіант завдання --- \studentnumber:
	\begin{enumerate}
		\item
			\begin{gather}
			\end{gather}
	\end{enumerate}

	\chapter{Теорія}
	\section{Завдання 1}
	\label{sec:prb1thr}

	\section{Завдання 2}
	\label{sec:prb2thr}

	\chapter{Про програму}
	Програма написана на \textinline|C++| з застосуванням Qt framework.

	Qt (вимовляється як \enquote{к'ют}) - це крос-платформне
	програмне забезпечення для створення графічних інтерфейсів користувача, а також
	крос-платформних додатків, які працюють на різних програмних і апаратних
	платформах, таких як Linux, Windows, macOS, Android або вбудованих системах, з
	невеликими змінами або без змін у базовій кодовій базі, залишаючись при цьому
	нативним додатком з нативними можливостями і швидкістю.

	\section{Про \texttt{C++}}
	\section{Файлова структура програми}
	Програма має наступну структуру:
	\begin{description}
		\item[\textinline{CMakeLists.txt}]
			Набір інструкцій \textinline|CMake| для побудови програми. 
		\item[\textinline{main.cpp}]
			Точка входу в програму. 
		\item[\textinline{calculator.cpp}]
			Імплементація графічного інтерфейсу користувача. 
		\item[\textinline|tabs.cpp|]
			Імплементація обчислень кожної вкладки. Вміст цього файлу зображений на \cref{fig:lst}.
	\end{description}
	До натиску кнопки \enquote{Claculate} в кожній вкладці прив'язана відповідна функція з \textinline{tabs.cpp}:
	\begin{funcDescription}
		\funcitem{cpp}|QString* tab1func(QString x, QString sqrtx, QString a, QString b, QString abyb)|
			Приймає рядки з введеннями користувача у вкладку 1, повертає буфер символів при
			вдалому обчисленні, або \cppinline|nullptr| при помилці. Блок-схема
			цієї функції зображена на \cref{fig:blockdiagram}.
		\funcitem{cpp}|QString* tab2func(QString a, QString aValue)|
			Приймає рядки з введеннями користувача у вкладку 2, повертає буфер символів при
			вдалому обчисленні, або \cppinline|nullptr| при помилці.
		\funcitem{cpp}|QString* tab3func(QString a)|
			Приймає рядки з введеннями користувача у вкладку 3, повертає буфер символів при
			вдалому обчисленні, або \cppinline|nullptr| при помилці.
	\end{funcDescription}
	
	\subsection{Алгоритм вирішення задачі 1}

	\subsection{Алгоритм вирішення задачі 2}

	\FloatBarrier
	\chapter{Тестування}
	Протестуємо отриманий прграмний застосунок. 
	\section{Розвʼязання задачі 1}
	Для розвʼязання задачі 1, введемо дані з умови в форму, та нажмемо кнопку \enquote{Compute}.
	\begin{figure}[h]
		\centering
		% \includegraphics[width=.35\textwidth]{\subfix{images/test1}}
		\caption{Розвʼязання задачі 1 за допомогою створеного програмного застосунку}%
		\label{fig:test1}
	\end{figure}
	\section{Розвʼязання задачі 2}
	Для розвʼязання задачі 2, введемо дані з умови в форму, та нажмемо кнопку \enquote{Compute}.
	\begin{figure}[h]
		\centering
		% \includegraphics[width=.32\textwidth]{\subfix{images/test2}}
		\caption{Розвʼязання задачі 2 за допомогою створеного програмного застосунку}%
		\label{fig:test2}
	\end{figure}

	\FloatBarrier
	\chapter{Висновки}
	В результаті розробки програмного застосунку було:
	\begin{itemize}
		\item 
		\item
	\end{itemize}
	Було отримано програмний застосунок який здатен:
	\begin{itemize}
		\item
		\item
		\item
	\end{itemize}
\end{document}
