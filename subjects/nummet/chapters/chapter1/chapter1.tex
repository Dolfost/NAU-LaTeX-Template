\documentclass[../../../../document]{subfiles}

\begin{document}
	\chapter{Постановка задачі}
	Вміст цієї секції взятий з~\cite{computational_methods}.\\
	\bigtext{Тема:} \worktheme.\\
	\bigtext{Мета:} 
	\begin{itemize}
		\item
	\end{itemize}
	Варіант завдання --- \studentnumber:
	\begin{enumerate}
		\item
	\end{enumerate}

	\chapter{Теорія}
	Вміст цього парагра опрацьовано з \cite{computational_methods} та лекцій.
	\section{Вирішення задачі 1}

	\FloatBarrier
	\chapter{Про програму}
	Програма написана на \textinline|C++| з застосуванням Qt framework.

	Qt (вимовляється як \enquote{к'ют}) - це крос-платформне
	програмне забезпечення для створення графічних інтерфейсів користувача, а також
	крос-платформних додатків, які працюють на різних програмних і апаратних
	платформах, таких як Linux, Windows, macOS, Android або вбудованих системах, з
	невеликими змінами або без змін у базовій кодовій базі, залишаючись при цьому
	нативним додатком з нативними можливостями і швидкістю.
	
	Обчислення математичних виразів у рядковому відображенні відбувається за допомогою бібліотеки \textinline|exptrk| яку написав Arash Partow:
	\begin{itemize}
		\item Офіційна сторінка автора: \url{https://www.partow.net/programming/exprtk/index.html}
		\item Сторінка проекту на \url{github.com}: \url{https://github.com/ArashPartow/exprtk/tree/master}
	\end{itemize}

	\section{Про \texttt{C++}}
	\subsection{Файлова структура програми}
	Програма має наступну структуру:
	\begin{description}
		\item[\textinline{CMakeLists.txt}]
			Набір інструкцій \textinline|CMake| для побудови програми. 
		\item[\textinline{main.cpp}]
			Точка входу в програму. 
		\item[\textinline{calculator.cpp}]
			Імплементація графічного інтерфейсу користувача. 
		\item[\textinline|tabs.cpp|]
			Імплементація обчислень. Вміст цього файлу зображений на \cref{lst:code}.
		\item[\textinline|exprtk_cmake/|] Бібліотека для обробки математичних виразів, як рядків.
	\end{description}

	\section{Алгоритм вирішення задачі}
	Було розроблено наступний алгоритм вірішення задачі:
	\begin{enumerate}
		\item Введення даних. 
		\item
		\item
	\end{enumerate}	

	Програмний код розробленого методу простих ітерацій зображений на \cref{lst:code1}.
	\begin{longlisting}
		 \begin{Center}
			 \inputminted{cpp}{\subfix{../../../../../include/iter.cpp}}
		 \end{Center}
		 \caption{Файл \textinline{iter.cpp}}\label{lst:code1}
	\end{longlisting}

	\FloatBarrier
	\chapter{Тестування}
	Протестуємо отриманий прграмний застосунок. 
	Правильність результатів роботи програми було перевірено на прикладах які наведені в \cite{computational_methods} у практичному занятті \prodnumber{}.
	\section{Контрольний приклад 1}
	Приклад 1 має наступну умову:

	Знайти додатні корені рівняння 
	\begin{gather}
		x^3-x-1=0.
	\end{gather}
	методом простої ітерації з точністю \(\varepsilon = 10^{-4}\). 

	Та наступну відповідь:
	\begin{gather}
		x = 1.3246\pm0.0001. 
	\end{gather}

	\begin{figure}[!h]
		\begin{center}
			\includegraphics[width=0.95\textwidth]{\subfix{images/test01}}
		\end{center}
		\caption{Розвʼязання контрольного прикладу 1}\label{fig:test01}
	\end{figure}
	Відповідь збігається з контрольним прикладом.

	\section{Розвʼязання задачі згідно варіанту}
	Було проведено розвʼязання задачі згідно варіанту. В результаті було отримано корені:
	\begin{gather}
		x_1=-1.49164,
	\end{gather}
	\begin{figure}[!h]
		\begin{center}
		\includegraphics[width=0.95\textwidth]{\subfix{images/test1}}
		\end{center}
		\caption{Розвʼязання задачі методом МПІ}\label{fig:test1}
	\end{figure}

	\FloatBarrier
	\newpage
	\chapter{Висновки}
	В результаті розробки програмного застосунку було:
	\begin{itemize}
		\item
		\item
		\item
	\end{itemize}
	Було отримано програмний застосунок який здатен:
	\begin{itemize}
		\item 
	\end{itemize}
\end{document}
